\chapter{Økonomi}
\label{chap:okonomi}
Hovedstyret er ansvarlig for foreningens økonomi i kraft av de vedtak som fattes. \newline

Leder av banKom skal legge frem status for linjeforeningens økonomi på hvert \linebreak hovedstyremøte samt gjøre det daglige økonomiarbeid for Hovedstyret.\newline

Økonomiansvarlig i øvrige komiteer er ansvarlig for å utarbeide budsjett og føre regnskap for sin komite. Gjennom året skal disse holde sin komite og banKom oppdatert på den økonomiske situasjonen og gjøre det daglige økonomiarbeidet for sin komite. På generalforsamlingen skal de økonomiansvarlige legge frem regnskap for det foregående året, samt legge frem budsjettforslag for neste år.\newline

Økonomiansvarlige kan ikke ta avgjørelser rundt fordeling eller forvaltning av \linebreak budsjeterte midler til ulike prosjekter i regi av linjeforeningen, disse avgjørelsene tas av den spesifikke komite eller Hovedstyret.\newline

Ikke-budsjetterte utgifter må godkjennes av Hovedstyret. Refundering av disse \linebreak utgiftene vil kun forekomme dersom utgiften er godkjent.\newline

Linjeforeningen kan søke støtte der den finner det hensiktsmessig, såfremt det ikke anses skadelig for linjeforeningens omdømme.\newline

%----
\section{Utilgjengelighet av økonomiansvarlig}

Om leder av banKom blir varig utilgjengelig, erklært uegnet ved mistillitsforslag i Hovedstyret, eller trer av, vil nestleder i banKom midlertidig ta over ansvaret for linjeforeningens økonomi. Hovedstyret plikter å avklare hvem som tar over ansvaret for linjeforeningens økonomi innen 14 dager etter at utilgjengeligheten av forrige leder for banKom blir sikker. 

%----
\section{Forsikringer}
Bank- og økonomikomiteen skal etterstrebe å holde linjeforeningen med gyldige forsikringer for midler og materiell som er disponert av linjeforeningen. Dette gjelder kun dersom bank- og økonomikomiteen mener at linjeforeningen har tilstrekkelige midler til at dette er fornuftig.
