\chapter{Generalforsamlingen}

Generalforsamlingen er linjeforeningens øverste organ og er uavhengig av gjeldende hovedstyrevedtak. Generalforsamlingen avholdes årlig i løpet av vårsemesteret.\newline

Den ordinære generalforsamlingen skal behandle årsmelding, innsendte saker, vedtektsendringer, valg og regnskap for foregående år. Valgkomite må velges for det neste år.

Hovedstyret kan i etterkant av Generalforsamlingen gjøre redaksjonelle endringer i vedtektene.  

%----
\section{Frister}
\label{sec:frister}
\begin{liste}
	\item Innkalling skal sendes ut til medlemmene senest \emph{fire (4) uker}  før \mbox{generalforsamlingen} skal avholdes.
	\item Saksforslag og forslag til vedtektsendringer sendes Hovedstyret senest \emph{to (2) uker} før generalforsamlingen skal avholdes.
	\item Fullstendig saksliste med vedtektsendringer skal tilgjengeliggjøres senest en (1) uke før møtedato. Denne skal også inneholde årsmelding, revidert regnskap, vedtatt budsjett for året og eventuelle andre relevante sakspapirer.
	\item Referat fra generalforsamlingen skal underskrives av paraferer og sendes \linebreak medlemmene eller gjøres tilgjengelig for medlemmene senest 14 dager etter generalforsamlingen.
\end{liste}


%----
\section{Ekstraordinær generalforsamling}
\vspace{23pt}
Denne kan innkalles av Hovedstyret eller om det minste av 1/8 av medlemmene og ti (10) medlemmer ønsker det. Fristene for å kalle inn til ekstraordinær generalforsamling er halvert i forhold til fristene for ordinær generalforsamling, jfr. 3.1.\newline

Ekstraordinær generalforsamling skal kun behandle den (de) saken(e) som står på dagsorden for den ekstraordinære generalforsamlingen
 

%----
\section{Organisering} \label{sec:organisering}
\vspace{23pt}
Ved generalforsamling er disse vervene nødvendig: 

\begin{liste}
	\item Ordstyrer
	\item To referenter - skriver referat under generalforsamling og samarbeider om \mbox{renskriving}
	\item Minst to til tellekorps - teller opp stemmer ved avstemming
	\item To paraferer - godkjenner referat fra generalforsamling og de endrede vedtektene i etterkant av generalforsamlingen
	\item Tre valgkomitémedlemmer - har et ansvar for å foreslå kandidater til neste års generalforsamling
	
\end{liste}

%----
\newpage
\section{Beslutningsdyktighet og avstemming}
\vspace{23pt}

For at en generalforsamling skal være beslutningsdyktig må det laveste mellom 15 medlemmer og 1/5 av medlemmene ha møtt opp. 

\begin{liste}
	\item Alminnelig flertall er definert som mer enn 1/2 av de tilstedeværende med stemmerett, med unntak av blanke stemmer. 
	\item Kvalifisert flertall er definert som minst 2/3 av de tilstedeværende med stemmerett, med unntak av blanke stemmer.
\end{liste}


Blanke stemmer er ikke tellende, med mindre annet er spesifisert.

\subsection{Saker}

Alle saker på generalforsamlingen avgjøres ved alminnelig flertall, med unntak av vedtektsendringer. 

\subsection{Vedtekter}
\vspace{23pt}

Vedtektsendringer avgjøres med kvalifisert flertall. 


Dersom en sak har bred støtte i salen kan ordstyrer forsøke å ta en sak opp til votering ved akklamasjon. Det skal gis rom for å uttrykke sin misnøye ved å vise tegn innen rimelig tid. Enhver stemmeberettiget person tilstede kan kreve at det gjennomføres en fullstendig avstemning.

Verken forhåndsstemming eller fullmakter er tillatt å bruke ved avstemming.


%----
\section{Stemmeberettigelse og talerett}
\vspace{23pt}
Ethvert medlem av linjeforeningen har talerett ved generalforsamlingen.
Ethvert medlem av linjeforeningen som er tilstede når generalforsamlingen godkjenner stemmeberettigede har rett til å stemme.\newline

Medlemmer av linjeforeningen som ikke har mulighet til å møte i tide plikter å informere Hovedstyret og oppgi en tilstrekkelig grunn til forsinkelse. Generalforsamlingen kan vedta å gi disse personene stemmerett samtidig som stemmeberettigede godkjennes.

Generalforsamlingen kan vedta å gi medlemmer av linjeforeningen som kommer for sent, og ikke har informert om dette, stemmerett når vedkommende ankommer.


%----
\section{Gjennomføring av valg}{
\vspace{23pt}
Dersom det er mer enn en kandidat til et verv skal det avholdes anonymt valg for det aktuelle vervet. Man kan stemme “ingen” på valget. For å regnes som vinner av valget må en kandidat få over halvparten (1/2) av stemmene. Ved manglende flertall fjernes kandidaten med færrest stemmer og valget går inn i en ny runde.\newline
					
Ved manglende flertall på kandidat med flest stemmer og stemmelikhet på de med færrest stemmer vil det avholdes en fullstendig ny runde.\newline

Innehavere av verv sitter inntil endt generalforsamling hvor det er gjennomført et godkjent valg for det respektive vervet. Dersom generalforsamlingen ikke klarer å gjennomføre et valg må det kalles inn til ekstraordinær forsamling innen tre dager etter endt ordinær generalforsamling.\newline

Ved opptelling av hemmelig valg skal tellekorps sitte i salen.


	\subsection{Fraskrivelse av rett til å stille til valg} {
	Personer som er innstilt med et av følgende verv under Generalforsamlingen fraskriver seg sin rett til å stille til alle andre valg.
	\begin{liste}
		\item Ordstyrer
		\item Referent
		\item Tellekorps
	\end{liste}
	Med å stille til andre valg menes det at man ikke kan stille, eller bli nominert, til andre verv under Generalforsamlingen og valg til Hovedstyret \newline

    Sittende valgkomité kan heller ikke stille til valg i Hovedstyret

	}
}
